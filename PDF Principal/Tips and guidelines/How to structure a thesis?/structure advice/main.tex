% $Id: main.tex 1765 2011-10-05 11:06:47Z s.gonzalez $

\documentclass[twocolumn,british]{article}

% $Id: preamble.tex 1765 2011-10-05 11:06:47Z s.gonzalez $
% !TEX root = main.tex

% This file contains commands, acronyms, hyphenations, and other
% auxiliary definitions.

%----[ Packages ]---

\usepackage[T1]{fontenc}
\usepackage[utf8]{inputenc}
\usepackage{babel}
\usepackage{a4wide}
\usepackage{enumitem}

%----[ Configuration ]---

\setdescription{leftmargin=0pt,style=nextline}

%----[ Commands ]---


\endinput

Emacs config
------------
Local Variables:
mode: latex
TeX-master: "main"
End:


\title{Proposed Structure for a Graduation Thesis}
\author{RELEASeD lab}

\begin{document}

\maketitle

We strongly advise you to write your thesis completely in \LaTeX. It
will produce a document of significant higher quality of layout, and
you can easily base yourself on a template of students who made their
thesis last year.

\section*{Abstract}

Your thesis typically starts with a 1-page abstract of the thesis: it provides a general motivation, a description of the problem you are addressing, a brief sketch of your solution to that problem, and finally it highlights the main results and contributions of the work.

\section*{Acknowledgments}

You can put here all acknowledgements of people that supported you directly or indirectly in this graduation thesis (or in your studies). If you want to give a funny twist to this section, that's okay.

\section*{Introduction}

The introduction states the problem of the thesis and is typically subdivided in the following short sections:
\begin{description}

\item[Context] Sketch the context and domain of the work.

\item[Problem] Describe and define the problem that will be tackled in
  this thesis.

\item[Motivation] Motivate why this problem is a relevant one. Why is
  this problem important, complex, not yet solved, what are the
  shortcomings today. Optionally add a motivation why you have chosen
  this particular thesis topic or why this topic was proposed.

\item[Objectives] Highlight the intended objectives of your
  thesis. What is the goal of this work? (In the conclusion chapter
  you will come back to these objectives and discuss to what extent
  you managed to achieve those objectives.)

\item[Approach] Without going in full detail, present in general terms
  how you will solve the problem and achieve the objectives.

\item[Contributions] Explicitly highlight the major and minor research
  contributions of your work. These can be of various kinds but it is
  important to stress the \emph{novel} aspect; e.g.:
  \begin{itemize}
  \item identification or specification of a new relevant problem;
  \item proposal of a novel solution to an existing problem;
  \item new mathematical formalisms, definitions, theorems or proofs;
  \item design of a novel framework, system, language, etc.;
  \item a comparison (or survey) of existing theories, models,
    designs, systems or implementations in a novel way;
  \item first implementation of a designed system (or significantly
    improved implementation of an existing system);
  \item an empirical analysis, for example a study of the performance
    of an implemented system;
  \item confirming or validation of the correctness of someone else’s
    work (for the first time or in a novel way).
  \end{itemize}

\item[Roadmap] Finish this section by announcing how the remainder of
  your thesis document is structured, e.g.:
  \begin{quote}
    The remainder of this document is structured as follows: the next
    two chapters provide the necessary background material and report
    on related work. Then, the proposed solution to the problem is
    introduced, motivated, defined and worked out. The software
    architecture supporting the implementation of that solution is
    then explained, and exemplified. An experiment / validation / case
    study is then conducted in order to validate the
    solution. Finally, a conclusion delivers the main contributions of
    this research and we present some avenues of future work.
  \end{quote}
  Hint: Try to phrase that roadmap paragraph in concrete terms
  dedicated to your concrete problem and solution. If someone else can
  just copy paste it and put it in his graduation thesis, then you are
  probably not concrete enough.
\end{description}

\section*{Background Material}

In this section, introduce the background material needed for the
reader to understand the core chapters of your thesis. This includes
all relevant material to make the thesis “self-contained” so that
reader can understand the entire thesis without having to consult
other sources. For example if you use a certain algorithm, or if you
target a certain platform, or if you rely on a certain mathematical
theory, it may be useful to summary the essence of that material here,
so that the user can fall back on it if needed. Users that already
know this material should be able to safely skip this section. Do not
forget to explain clearly to the reader why this material is relevant
in the context of this thesis, so that he doesn’t get bored by all
this extra stuff, which is only background material but not part of
the essence of the thesis. Don’t exaggerate with this section. Only
put what is really necessary to understand what comes later. Some
theses don’t need a background section at all.

\section*{Related Work}

In this section you should list relevant related work existing today:
what other solutions are provided today to address the problem that
you stated in the introduction. You could: provide a summary of some
articles that you have read, make a comparison table between solutions
that you found relevant for the problem, analyze existing solutions,
discuss advantages and shortcomings of some selected solutions,
discuss in details the limitations of existing practice. The main
point of the related work is to put in perspective your work, which
will build upon or be complementary to that related work.

\section*{Problem Statement}

Now that you have introduced the necessary background material and
related work, you can explain in more detail the research problem that
you want to address in this thesis, why it is a relevant problem and
how your solution would advance the state of the art in this research
area (by positioning it in terms of the related work discussed in the
previous chapter).

\section*{Running Example [optional]}

When explaining your solution in the next section, it is often helpful
to do that through a running example. That running example can be
introduced here. It should be well chosen so that it is not too
complex yet illustrates the problem well. Note that this running
example does not have to be a full-fletched case study – that would be
part of the validation chapter – but can be smaller. Alternatively, it
could be a subpart of such a larger case study.

\section*{Solution}

In this chapter (or chapters), you describe in detail the solution you
have developed to address the problem. This chapter may be decomposed
in several chapters describing different aspects of your solution. For
example, one chapter introducing a new formalism you developed,
another describing a novel algorithm you propose based on that
formalism, and finally a chapter discussing a prototype implementation
of that algorithm.

What is most important to make clear in these chapters is the
conceptual ideas behind the solution. Try to describe the essence of
your solution so that any reader can understand it, even though you
intricacies of the solution as well. A reader who is only interested
can add some more detailed sections that dive into the technical in
the big picture should be able to skip those more detailed sections
and still understand what your solution is about. A reader who wants
to understand your solution in full detail can decide to read them.

Where necessary, provide schemas or pictures illustrating how your
solution works. A (good) picture often tells more than a thousand
words. Throughout this chapter, illustrate the different aspects of
your solution on the running example. At the end of the chapter, don’t
forget to position your particular solution to the related work
discussed in a previous chapter.

\section*{Validation}

In this chapter, describe the experiment, benchmarks, case study or
other means of validation, which you conducted to prove that your
solution attains the objectives put forward in the introduction.

\begin{description}

\item[Method] Describe the details and approach of how you conducted
  your validation experiment(s).

\item[Results] Present the brute results obtained after having
  conducted the experiment, but don’t draw any conclusions yet.

\item[Analysis/discussion] Analyse the obtained results and discuss
  what conclusions you can draw from these results. If possible
  include statistical tests, charts, graphs to support your analysis.

\item[Threats to validity] Discuss all factors that may have
  negatively or positively influenced your results or that may cause
  the experiment to be difficult to replicate by others.

\item[Conclusion] Summarize the main findings of your work: what did
  you do, what was not covered, advantages, shortcomings, possible
  future work. Did you attain the initial objectives of the thesis?

  Make sure to revisit the initial problem statement and to point out
  explicitly how your solution addresses it. Also repeat the achieved
  contributions.

\end{description}

\section*{Future Work}

Discuss possible paths of future work here. A good idea is to fill in this section throughout the entire year you work on this thesis. Whenever you have a cool idea, put it here. If you don’t have time to develop it, it becomes future work.

\section*{References}
A section with all references to books, articles and web-sites you
refer to in the thesis. Make your that your references are as complete
and detailed as possible and let BibTex take care of the formatting
for you.

\end{document}

\endinput

Emacs config
------------
Local Variables:
TeX-master: t
End:
