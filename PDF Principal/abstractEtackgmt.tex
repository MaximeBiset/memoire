\newpage
\thispagestyle{empty}
\mbox{}

\section*{Abstract}

Ce mémoire a pour but de mettre en place un outil en alliant 2 domaines particuliers dans leur discipline respective.  D'une part,  ce mémoire se démarque des projets informatiques plus classiques car il a pour but de développer un framework.  Ce type de logiciels est destiné à servir de base pour être appliqué à plusieurs cas concrets.   Ainsi,  pour obtenir une telle solution,  l'analyse du domaine ainsi que le développement se font selon des méthodes spécialement concues pour ce cas de figure.  L'objectif est à chaque fois de pouvoir analyser et distinguer les parties communes des parties spécifiques et de faire en sorte que ces dernières soient facilement adaptables.  D'autre part,  le domaine dans lequel se déroule ce projet informatique est particulier aussi car il s'agit d'économies qui ont pour but de favoriser les échanges.  Cette particularité est importante pour l'utilisabilité du logiciel final.

Ce mémoire apporte plusieurs résultats pour répondre à la demande d'un framework pour ce type d'économies.  Tout d'abord,  le framework développé en lui-même.  Celui-ci permet d'être adapté selon les réalités de l'économie pour lequel l'outil est destiné.  Mais le framework est le résultat d'un long cheminement qui a produit d'autres éléments intéressants.  D'abord,  l'analyse réalisée donne un bon aperçu des outils qui peuvent être utilisés ainsi que de leurs fonctionnalités.  Ensuite,  le développement du framework au sein de Django a requis d'utiliser et rassembler diverses techniques qui peuvent être réutilisées pour élargir le framework pour qu'il offre de nouvelles possibilités.

%%%%%%%%%%%%%%%%%%%
\newpage
\thispagestyle{empty}
\mbox{}
%%%%%%%%%%%%%%%%%%%

\section*{Acknowledgments}

Je tiens à remercier toutes les personnes qui m'ont aidées et soutenues pour la réalisation de ce mémoire.  En particulier mon promoteur,  Kim Mens,  pour ses nombreux conseils,  explications et relectures ainsi que Katleen Deruytter pour toutes ses explications et  suggestions.  Je remercie également les étudiants du groupe du cours Software Engineering Project (Denis Genon, Thibault Gerondal, Michaël Heraly, Baptiste Lacasse, Arnold Moyaux, Aloïs Paulus, Jérémy Vanhee et Victor Velghe) pour leur disponibilité.  

Enfin,  j'espère que ce mémoire permettra de voir apparaître plus d'économies locales de partage,  ou d'améliorer les existantes,  afin d'offrir au monde de demain des alternatives au système actuel.
\newpage
\thispagestyle{empty}
\mbox{}