Contexte du mémoire
*********************
Ce mémoire s'est déroulé dans un domaine assez particulier et vaste.  En effet,  BuurtPensioen fait partie de ces initiatives qui sortent quelques peu du schéma classique "commerçant-client".  Dans cette relation traditionnelle,   l'un des 2 acteurs exerce une profession incluant la vente de biens et/ou services que le second acteur achète.  Dans notre cas,  l'objectif est souvent de mettre en relation 2 acteurs pour une transaction plus unique et qui n'implique pas d'offce que l'un des deux acteurs soit plus "professionnel" que l'autre.  




Déroulement de l'analyse.

La demande concernait le développement d'une application open-source De plus, le client n'avait pas d'idée sur les fondements juridiques de ce type de logiciels.  Une première étape a donc été d'analyser les licences existantes pouvant être appliquées à ce projet.  A cet effet,  j'ai consulté le service Administration de la Recherche de l'UCL.  Un entretien avec Sébastien Adam m'a permi d'éclaircir ce domaine.  A partir de là,  j'ai réalisé une petite note résumant les possibilités pour ce projet afin que le client puisse choisir la plus adéquate.  

*************************************
EXPLICATION LICENCES OPEN SOURCE
**************************************

Une fois cette question réglée,  une rencontre avec le client et d'autres potentiels partenaires du projet a eu lieu.  Ce fut l'occasion d'avoir un meilleur aperçu de différents cas similaires à BuurtPensioen (autres villes,  projets un peu différents) ainsi que des solutions existantes et/ou en cours de développement.  Le schéma ci-après résume assez bien la situation actuelle concernant les acteurs et outils existants dans le domaine des économies d'échange de personne à personne.  

****************************
SCHEMA ACTEURS ET OUTILS
*****************************

Acteurs : 
- BuurtPensioen

-> Utilise des documents de bureautique,  centralisés et gérés par quelques personnes.  La communication entre les "administrateurs" et les utilisateurs se fait principalement par téléphone.
- 
- 
- 

Troeven.be : 
https://www.troeven.be/nl-be/faq/

Time2Care : 
http://www.time2care.be/
