\section{Approche}

Pour faire face au problème décrit dans le point précédent,  une méthode en plusieurs étapes est nécessaire.  
Tout d'abord,  comme dans la plupart des projets informatiques,  il est important de réaliser une analyse du domaine.  Cependant,  une analyse du domaine "classique" ne conviendrait pas pour le problème tel que décrit ci-dessus.  En effet,  l'objectif étant de réaliser un logiciel applicable à plusieurs organisations ayant chacune ses particularités,  l'analyse doit être faite dans le but de développer,  par la suite,  un framework.  Pour cela,  on fera d'abord un survol du domaine en définissant les principaux concepts puis  nous réunirons un maximum d'informations dans un feature diagram.  Celui-ci permettra de mettre en avant les différentes composantes et possibilités du système et de son outil.  De plus,  ce diagramme est accompagné de contraintes entre les différentes composantes.  Ceci permet d'éviter de définir une application dont certaines parties ne sont pas compatibles entre elles.  Enfin,  lors du développement,  le feature diagram est une bonne source d'organisation.  En effet,  certains features pourront correspondre à des parties "indépendantes".  Cette étape peut donc amener déjà des repères à l'instar de l'étape du design dans une analyse plus traditionnelle.
Ces étapes d'analyse amènent une bonne compréhension du domaine dans lequel nous travaillons ainsi qu'un premier pas vers la construction d'un framework qui devrait répondre aux besoins des différents cas existants.  

La seconde étape pour résoudre le problème posé va consister à partir de l'analyse réalisée et du logiciel choisi comme base de départ,  pour développer le framework qui devra répondre aux exigences définies.  Dans la pratique,  un framework est rarement développé "from scratch",  c'est-à-dire à partir de rien.  Généralement,  un ou plusieurs outils existent déjà et ceux-ci sont analysés et refactorés afin de produire une version plus générique de l'applicaiton.  Nous avons déjà abordé le choix de l'application de base lors de l'explication du problème.  Une fois cette base choisie,  il faut se réapproprier le résultat du travail du groupe d'étudiants et analyser l'architecture ainsi que les fonctionnalités et leur implémentation.  Ensuite,  il faut se nourrir d'autres cas qui n'étaient pas prévu à la base dans le logiciel développé afin de les intégrer au framework.

La dernière étape pour répondre au problème posé consiste à valider la solution proposée.  Pour cela,  nous allons prendre 2 cas suffisamment opposés et essayer de les implémenter au moyen du framework produit.  Le premier cas sera BuurtPensioen avec les fonctionnalités développées par le groupe choisi au début du développement.  Le second cas sera.... 

TO DO : CONTINUE ! !