\section{Future Work}

Suite aux constats faits lors de la validation,  une première suite qui peut être donnée à ce projet consiste à continuer de débugger le framework afin qu'il soit fonctionnel à 100\%.  L'étape suivante peut consister en l'ajout de nouveaux features.  La méthode pour y parvenir devrait,  selon moi,  être un développement incrémental,  tel que commencé pour ce projet. En effet,  il est plus facile de suivre la démarche que nous avons décrite dans le chapitre sur le développement et ce,  pour chaque feature un par un.  Ceci semble obligatoire à la vue de la complexité du projet.  

D'une manière de plus globale,  il peut être intéressant d'envisager le concept de communauté de développeurs.  En effet,  ce framework est à destination d'organisations que l'on peut qualifier de ``transitionnaires'' et il a été développé sous licence open-source.  Ainsi,  ce serait une belle opportunité qu'un groupement se crée autour du framework afin d'une part pouvoir soutenir les organisations locales qui désirent instancier le framework,  et d'autre part,  développer de nouveaux features.  Il s'agit là,  selon moi,  d'une piste intéressante pour que ce projet soit utile à la société et utilisé par un maximum de projets locaux.  

